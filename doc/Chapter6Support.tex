\chapter{Support}
\section{Issues and Features}

Google code provides a bug tracker on project pages. Users and developers can write new issues or features, called tickets, at \href{http://code.google.com/p/adaptive-sampling-kit/issues/list}{ASK bug tracker}. To open tickets please follow the rules below.

\subsection{Program Bugs}

The following steps should be used to open a bug ticket:

\begin{enumerate}
	\item Template: Choose Defect request from user or developer, accordingly.
	\item Subject: Choose an explicit title, summarizing the issue.
	\item Description: Add a brief abstract of the bug. Describe the exact steps that triggered the bug, and if possible attach the files needed to reproduce it.
\end{enumerate}

\subsection{Feature Requests}

The following steps should be used to open a feature ticket:

\begin{enumerate}
	\item Tracker: Choose ``Review request.''
	\item Subject: Choose an explicit title, summarizing the issue.
	\item Description: Add a description of the feature.
\end{enumerate}

\section{Contributing}

The project is particularly interested in:
\begin{itemize}
	\item Patches fixing bugs or adding new features
	\item Documentation fixes
	\item Bug reports
\end{itemize}

For documentation fixes and bug reports, please use the \href{http://code.google.com/p/adaptive-sampling-kit/issues/list}{issue tracker}.
For code contributions, contact us at \textbf{ask-team@exascale-computing.eu}.

\subsection{Running the Test Suite}

ASK comes with an extensive suite of tests. Before submitting a patch, please make sure that all the tests pass, with the following command:
\begin{minted}{bash}
cd tests/
nosetests -v
\end{minted}

If you submit a patch, including unit tests for your code is more than welcome and will make the patch review faster.
